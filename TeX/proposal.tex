\documentclass[11pt]{article}
\usepackage{amsmath}
\usepackage{graphicx}
\graphicspath{ {/Users/yvesmarcel/Documents/sFDR_paper/assets/images/} }
\begin{document}
\title{Extending the Fitzhugh Nagumo Model of Relaxation Oscillators to Model Cardiac Necrosis}
\author{Aaron R. Shifman $^{1,2}$, Christopher B. Cole $^{2,3}$}
\maketitle
\tableofcontents
%\tableoffigures

\begin{abstract}

Neuroscience frequently relies on robust mathematical models to simulate and investigate trends in neuromechanics and signal propagation. We propose an examination of the Fitzhugh-Nagumo model of relaxation oscillators and additionally propose a modified extension of this model to account for necrosis of cardiac cells. We use this model to investigate case studies in cardiac necrosis and examine the effect of necrosis distribution on a one dimensional system using a propagation model with boundary conditions. 
\end{abstract}

\section{Examination of the Fitzhugh Nagumo Model}

The modelling of neurons and associated action potentials started in the early 1900s with the \"integrate and fire\" model on analogue computers. In 1952, Hodgkin and Huxley created a biologically relevant 4 dimensional system of equations to describe neuron dynamics at a point.$^1$ While very interesting to biologists, mathematicians found the model too difficult to systematically study. In 1961 Richard Fitzhugh created a 2D caricature of the Hodgkin Hudley model (activation and recovery) using a modified van-der-pol oscillator. This model has been both well studied and well characterized. Additionally, it has found many uses in cardiac models due to its simplicity and similarity in shape to a cardiac action potential. 

The Hodgkin Huxley equations described how neuronal membrane voltage changed over time, given a large number of parameters. 

\begin{align}
C_m \frac{dV}{dt} &= I_m = I_{Na} + I_{K} + I_l \\
I_x &= \bar{G_x}p_{open}(V_m-E_x); x\in\{Na,K,l\}\\
p_{open_{K}} &= n^4\\
p_{open_{Na}} &= m^3h\\
p_{open_{l}} &\equiv 1\\
\frac{dx}{dt} &= \alpha_x(V)(1-x)-\beta_x(V)x; x\in{m,n,h}\\
\end{align}
This is impossible to solve analytically, and quite difficult to explore numerically due to the high dimensionality of the system and the large number of parameters. In studying hyperplanes through the 4D phase space, Nagumo noticed that all planes of activation and recovery variables had a similar shape, so he tried modeling neuronal activity as a modified Van-der-Pol oscillator (Bonheffer-Vanderpol oscillator). This model, while not looking like steroetyped action potentials, has several features of the full hodgkin huxley model such as activation hysteresis and spike blocking. Despite not looking like a neuronal action potential, it closely resembles cardiac action potentials and have since been adapted as one of the de-facto models for cardiac simulations. 

% <- maybe talk about calcium plateaeu 
%is there anything here to cite? i feel like we should cite stuff

\subsection{Model Rationale and Variables}

The system presents in the following, non-dimensionalized form. This is not the original Bonheffer-Vanderpol model, however it is topologically equivalent and much easier to analyse. More information in section 1.4.

\begin{align} \frac{dV}{dt} &= V(a+V)(1-V) -W + z \\
\frac{dW}{dt} &= bV-cW \end{align}

In this model, $\chi$ - in general - represents activation, and in an excitable tissue context represents voltage. and $\gamma$ - in general represents recovery, and has no realistic interpretation in a biological context. There are three parameters in the model, $\alpha$, $\beta$, and $z$. Of the three parameters, we have significant interest in $z$, as it will determine the bifurcation of the model. Neurologically, $z$ represents the DC offset from experimentally injected current of the neuron under study. In our extension to this model, we will alter $z$ proportion to the dampening of current between neurons. The additional parameters $\alpha$ and $\beta$ represent rate - like constants which determine oscillator frequency and duty cycle, as well as excitability.


\subsection{Purpose of Model and Examination}

This model, a simplification of the original HH model, intends to model the spiking and relaxation of neurons. Tuning the individual parameters can adapt the model to fit various states and conditions which may be investigated by researchers. Various combinations of of parameters can also model various neuronal behaviors such as bursting and chattering. These models can be put together to form multi dimensional models of neuron trends. The excitation and relaxation of the neuron relies on the $z$ parameter to determine its stability, as discussed in the next section. 
In this project, we plan to evaluate the stability of system at rest and with regards to the bifurcation parameter $z$. We plan to evaluate how the system responds and bifurcates with regards to this parameter, and interpret these stabilities biologically. %cite the chattering probably

Additionally, we will examine and interpret the 

\begin{enumerate}
	\item Phase Plane
	\item Bifurcation Diagram
	\item Time Series
\end{enumerate}

The model itself is already non-dimensionalized, though in our analysis we will provide a full description of the author's rationale. Additionally, we will fully classify the model according to the Haefner et. al criteria and justify our classification.

%end original model, begin extension

\section{Modelling Cardiac Necrosis} 
\subsection{Necrosis Propagation}

As part of the normal cell cycle, programmed cell death (apoptosis) exists and it serves several purposes. In most cases, it is both irreversable and adaptive (benificial) for the organism. However in some cases, apoptosis can occur accidentally in which case an irreversible destruction of the cell will occur. In our model, we assume that necrosis is a function of an underlying accumulation of pro-apoptotic proteins, potentially introduced as a by product of disease. As such, necrosis is a continuous state of each cell. For cell $i$ of $n$, define pro-necrotic burden as $\nu_i$, with total burden of the system equal to $\frac{\sum^n_{i=1} \nu_i}{n}=\frac{1}{n} \sum^n_{i=1} \nu_i$. Note that $\nu_{1}, ..., \nu_{n} \in [0,1]$ and is a monotonically increasing function with respect to time, measured in neuronal cycles. 

\subsection{Investigation Goals and Expected Results}

In order to model the heart and disease models, we will couple cardiac myocytes (with gap junctions) to form a one dimensioanl caricature of cardiac electrical propagation. Gap junctions act by physically linking 2 cells, i.e. the current will be proportional to the electrical potential between the 2 cells. As such, a spike in one cell can cause a spike in another cell. By having the final cell feedback onto the first cell, we can form a very accurate model of a heart.

We want to ask how the temporal effects and distribution of cardiac death effects the overall frequency of the network. We expect that each diseased cell will act as a delay line, in which if 2 cells have equal states of disease $\alpha$ and $\beta$, this will be equal to a single cell with a disease state $\alpha+\beta$. 

Define $D$, the delay caused by pro-necrotic built up, such that $D=f_n - f_t$ where $f_r$ represents the frequency of firing at rest and $f_t$ represents the frequency of firing at time $t$. 

Consider a system with two, gap junctionally connected, neurons $N_1$ and $N_2$. $D$ is a function such that $D(\nu_1 N_1, \nu_2 N_1)$ defines the system's delay caused by $\nu_1$ at $N_1$ and $\nu_2$ at $N_2$. It is our hypothesis: 

$$ \begin{aligned} \nu_1 &= \alpha \\ \nu_2 &= \beta \\ \nu_3 &= \beta + \alpha \\ \nu_4 &= 1 \\ D(\nu_1 N_1, \nu_2 N_1) &= D(\nu_3 N_1, \nu_4 N_1) \end{aligned} $$ 

To examine and support our hypothesis, we will computationally model the above model using the FN model and gap junctioned neurons. We will construct graphics that demonstrate and illustrate our hypothesis.

Additionally, we will explore how the system responds in a variety of circumstances:

\begin{enumerate}
	\item Does distance (Distance between $N_i$ and $N_j= j -i$) influence $D$
	\item Is $D$ decreased more by a few heavily burdened cells, or by a large number of moderately burdened cells
	\item At what point does the build up of pro-necrotic proteins become a health issue
\end{enumerate}

The above will be examined in both a static ($\nu_1$ remains contant with time) and a dynamic ($\nu_i$ is a function of $t$) model of cardiac necrosis. 

\subsection{Translational Perspective} %how the model relates to disease

The heart serves to transport blood - an essential function for life, in other words, should the heart fail to deliver a the requisite amount of blood, the organism will die. What we have done here is to create a model where we look at “heart rate” as a function of disease, and we can evaluate death (some threshold frequency) as a function of disease and distribution as a first pass into understanding cardiac death.

\section{Discussion of Report} % (fold)
\label{sec:discussion_of_report}

\subsection{Proposed Format} % (fold)
\label{sub:proposed_format}

We propose the following format for the final report:

\begin{enumerate}
	\item Introduction
	\item Primary Model Analysis
	\begin{enumerate}
		\item Variable Description
		\item Model Classification
		\item Time series 
		\item Phase Plane
		\item Orbit / Bifurcation Diagrams
	\end{enumerate}
	\item Cardiac Necrosis Extension
	\begin{enumerate}
		\item Description of extension
		\item Variable Explaination
		\item Time series with incorporation of $\nu$
		\item Computational Model Rational
		\item Static $\nu$
		\begin{enumerate}
			\item Distance
			\item Differential burden of System 
			\item Interpretation
		\end{enumerate}
		\item Dynamic $\nu$
		\begin{enumerate}
			\item Distance
			\item Differential burden of System 
			\item Interpretation
		\end{enumerate}
	\end{enumerate}
	\item Discussion
	\item Conclusions and Further Work
	\item References
	\item Appendix I: Code 
\end{enumerate}


\subsection{Timeline} % (fold)
\label{sub:timeline}

We propose a checkup in two weeks time to assess our progress and make sure that the project presented is reasonable in scope for the course. If possible, we would like to present a draft to ensure that we are on the right path. 

% subsection timeline (end)

% subsection proposed_format (end)

% section discussion_of_report (end)

\newpage

\section{References} % (fold)
\label{sec:references}

\begin{enumerate}
	\item HODGKIN, A. L., HUXLEY, A. F. A quantitative description of membrane current and its application to conduction and excitation in nerve. J. Physiol. 117, 500–44 \(1952\).
	\item Haefner, J. W. Models of Systems. Modeling Biological Systems: Principles and Applications \(Springer, 2005\). doi:10.1007/0-387-25012-3\_1
\end{enumerate}


% section references (end)

\end{document}


